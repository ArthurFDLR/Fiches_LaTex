\documentclass[french]{article}
 
\usepackage[top=2cm, bottom=2cm, left=2cm, right=2cm]{geometry}
\usepackage[utf8]{inputenc}
\usepackage[T1]{fontenc}
\usepackage{babel}
\usepackage{inputenc}
\usepackage{verbatim}
\usepackage{fancyvrb}
\usepackage{enumitem}
\usepackage{setspace}
\usepackage{ulem}
\usepackage{amsmath}
\usepackage{amsthm}
\usepackage{amssymb}
\usepackage{graphicx}
\usepackage{eso-pic}
\usepackage{graphics}
\usepackage{xcolor}
\usepackage{caption}
\usepackage{makecell}
\usepackage{eurosym}
\usepackage{textpos}
\usepackage{pdfpages}
\usepackage{tocloft}

\usepackage{titlesec}
\titleformat*{\section}{\LARGE\bfseries}
\titleformat*{\subsection}{\Large\bfseries}
\titleformat*{\subsubsection}{\large\bfseries}

%Saut de ligne%
\newcommand{\bl}[0]{\vspace{1\baselineskip}}
\newcommand{\blo}[1]{\vspace{#1\baselineskip}}
%Numérotation des sections romaine%
\renewcommand{\thesection}{\Roman{section}} 
%Affichage des numéros de section dans les figures%
\renewcommand{\thefigure}{\thesection.\arabic{figure}}

\theoremstyle{definition}
\newtheorem{definition}{Definition}[section]

\begin{document}
\begin{center}
	\textbf{\Large{Mécanique des fluides - S3}}
\end{center}

\section{Equations de bilans}
\begin{definition}Bilan de masse (continuité)\par
$$ \cfrac{\partial \rho}{\partial t} + div(\rho \vec{v}) = 0 $$
	On en déduit l'équivalence : écoulement incompressible $ \leftrightarrow div(\vec{v})$
\end{definition}

\begin{definition}Bilan de quantité de mouvement\par
	$$ \rho \frac{d \vec{v}}{dt} = -grad(p) + \mu \Delta\vec{v} + \vec{f} $$
	en développement la dérivé droite et en introduisant la viscosité cinématique ($\nu = \frac{\mu}{\rho}$) :
	$$ \frac{\partial \vec{v}}{\partial t} + \bar{\bar{grad}}.\vec{v} = -\frac{grad(p)}{\rho} + \nu \Delta\vec{v} + \frac{\vec{f}}{\rho} $$
\end{definition}

\begin{definition}Loi de Newton\par
	\begin{itemize} 
		\item Tenseur de contraintes : $\bar{\bar{\sigma}} = -p* \bar{\bar{I}} + \bar{\bar{\tau}}$
		\item Tenseur de contraintes visqueuses : $ \bar{\bar{\tau}} = 2 \ \mu \ \bar{\bar{D}} + \eta \ div(\vec{v}) \ \bar{\bar{I}} = 2 \ \eta \ (\bar{\bar{D}} - \frac{div(\vec{v})}{3} \bar{\bar{I}}) $
		\item Tenseur du taux de déformation : $ \bar{\bar{D}} = sym(\bar{\bar{grad}}(\vec{v})) $
		\item Taux de variation de volume : $div(\vec{v}) = tr(\bar{\bar{D}})$
		\item Hypothese de Stockes sur les coéfficients de viscosité (viscosité de volume nulle) : $ 2 \mu + 3 \eta =0$
	\end{itemize}
	On en déduit la force totale appliqué sur un solide immergé : $ \vec{F} = \int_S \bar{\bar{\sigma}} \vec{n} \ dS $
\end{definition}

\section{Analyse des écoulements incompressibles}
\subsection{Echelles caractéristiques}

\begin{definition} Temps caractéristiques\par
	\begin{itemize} 
		\item Temps caractéristiques de transport \textbf{advectif} (transport sur L à la vitesse V) : $T_a = \frac{L}{V}$
		\item Temps caractéristiques de transport \textbf{diffusif} (affectation de la ligne $\sigma$ par la viscosité) : $T_d = \frac{\sigma^2}{\nu}$
	\end{itemize}
	Avec \textbf{L} la distance caractéristiques, \textbf{T} le temps caractéristiques et \textbf{V} la vitesse caractéristiques.
\end{definition}

\begin{definition} Nombres caractéristiques\par
	\begin{itemize} 
		\item \textbf{Nombre de Strouhal} : Caractéristise l'instationarité
		
		$St = \frac{T_a}{T} = \frac{L}{VT}$
		$
		\left\{
			\begin{array}{ll}
				St << 1 \rightarrow \mbox{quasi-permanent} \\
				St \approx 1 \rightarrow \mbox{instationnaire} \\
				St << 1 \rightarrow \mbox{fortement instationnaire} \\
			\end{array}
		\right.
		$

		\item \textbf{Nombre de Reynolds} : Caractéristise les effets visqueux (matériaux géologique, bactéries, lubrifiaction \dots)
		
		$Re = \frac{T_d}{T_a} = \frac{VL}{\nu}  \approx \frac{\mbox{force d'inertie}}{\mbox{force de viscosité}}$
		$
		\left\{
			\begin{array}{ll}
				Re << 1 \rightarrow \mbox{ecoulement rampant} \\
				St \approx 1 \rightarrow \mbox{diffusion et advection du même ordre de grandeur} \\
				St << 1 \rightarrow \mbox{effets de viscosité négligeables} \\
			\end{array}
		\right.
		$
		\item \textbf{Nombre de Froude} : $St = \frac{V^2}{L \ g}  \approx \frac{\mbox{force d'inertie}}{\mbox{force de pesanteur}}$ 
	\end{itemize} 
\end{definition}

\begin{definition} Couche limite (approfondie par la suite) \par
	Distance à la paroie, notée $\sigma$, sur la quelle le milieu est affectée par la diffusion. 
	$$ \sigma \approx \sqrt{\nu \ T_a } \approx \sqrt{\nu \ T_d }$$ 
\end{definition}

\subsection{Rotation dans les écoulements}


\end{document}