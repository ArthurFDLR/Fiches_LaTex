\documentclass[french]{article}
 
\usepackage[top=2cm, bottom=2cm, left=2cm, right=2cm]{geometry}
\usepackage[utf8]{inputenc}
\usepackage[T1]{fontenc}
\usepackage{babel}
\usepackage{inputenc}
\usepackage{verbatim}
\usepackage{fancyvrb}
\usepackage{enumitem}
\usepackage{setspace}
\usepackage{ulem}
\usepackage{amsmath}
\usepackage{amsthm}
\usepackage{amssymb}
\usepackage{graphicx}
\usepackage{eso-pic}
\usepackage{graphics}
\usepackage{xcolor}
\usepackage{caption}
\usepackage{makecell}
\usepackage{eurosym}
\usepackage{textpos}
\usepackage{pdfpages}
\usepackage{tocloft}

\usepackage{titlesec}
\titleformat*{\section}{\LARGE\bfseries}
\titleformat*{\subsection}{\Large\bfseries}
\titleformat*{\subsubsection}{\large\bfseries}

%Saut de ligne%
\newcommand{\bl}[0]{\vspace{1\baselineskip}}
\newcommand{\blo}[1]{\vspace{#1\baselineskip}}
%Numérotation des sections romaine%
\renewcommand{\thesection}{\Roman{section}} 
%Affichage des numéros de section dans les figures%
\renewcommand{\thefigure}{\thesection.\arabic{figure}}

\theoremstyle{definition}
\newtheorem{definition}{Definition}[section]

\begin{document}
\begin{center}
	\textbf{\Large{Mécanique des fluides - S3}}
\end{center}

\section{Equations de bilans}
\begin{definition}Bilan de masse (continuité)\par
$$ \cfrac{\partial \rho}{\partial t} + div(\rho \vec{v}) = 0 $$
	On en déduit l'équivalence : écoulement incompressible $ \Leftrightarrow div(\vec{v})$
\end{definition}

\begin{definition}Bilan de quantité de mouvement\par
	$$ \rho \frac{d \vec{v}}{dt} = -grad(p) + \mu \Delta\vec{v} + \vec{f} $$
	en développement la dérivé droite et en introduisant la viscosité cinématique ($\nu = \frac{\mu}{\rho}$) :
	$$ \frac{\partial \vec{v}}{\partial t} + \bar{\bar{grad}}(\vec{v}).\vec{v} = -\frac{grad(p)}{\rho} + \nu \Delta\vec{v} + \frac{\vec{f}}{\rho} $$
\end{definition}

\begin{definition}Loi de Newton\par
	\begin{itemize} 
		\item Tenseur de contraintes : $\bar{\bar{\sigma}} = -p* \bar{\bar{I}} + \bar{\bar{\tau}}$
		\item Tenseur de contraintes visqueuses : $ \bar{\bar{\tau}} = 2 \ \mu \ \bar{\bar{D}} + \eta \ div(\vec{v}) \ \bar{\bar{I}} = 2 \ \eta \ (\bar{\bar{D}} - \frac{div(\vec{v})}{3} \bar{\bar{I}}) $
		\item Tenseur du taux de déformation : $ \bar{\bar{D}} = sym(\bar{\bar{grad}}(\vec{v})) $
		\item Taux de variation de volume : $div(\vec{v}) = tr(\bar{\bar{D}})$
		\item Hypothese de Stockes sur les coéfficients de viscosité (viscosité de volume nulle) : $ 2 \mu + 3 \eta =0$
	\end{itemize}
	On en déduit la force totale appliqué sur un solide immergé : $ \vec{F} = \int_S \bar{\bar{\sigma}} \vec{n} \ dS $
\end{definition}

\section{Analyse des écoulements incompressibles}
\subsection{Echelles caractéristiques}

\begin{definition} Temps caractéristiques\par
	\begin{itemize} 
		\item Temps caractéristiques de transport \textbf{advectif} (transport sur L à la vitesse V) : $T_a = \frac{L}{V}$
		\item Temps caractéristiques de transport \textbf{diffusif} (affectation de la ligne $\sigma$ par la viscosité) : $T_d = \frac{\sigma^2}{\nu}$
	\end{itemize}
	Avec \textbf{L} la distance caractéristiques, \textbf{T} le temps caractéristiques et \textbf{V} la vitesse caractéristiques.
\end{definition}

\begin{definition} Nombres caractéristiques\par
	\begin{itemize} 
		\item \textbf{Nombre de Strouhal} : Caractéristise l'instationarité
		
		$St = \frac{T_a}{T} = \frac{L}{VT}$
		$
		\left\{
			\begin{array}{ll}
				St << 1 \Rightarrow \mbox{quasi-permanent} \\
				St \approx 1 \Rightarrow \mbox{instationnaire} \\
				St >> 1 \Rightarrow \mbox{fortement instationnaire} \\
			\end{array}
		\right.
		$

		\item \textbf{Nombre de Reynolds} : Caractéristise les effets visqueux (matériaux géologique, bactéries, lubrifiaction \dots)
		
		$Re = \frac{T_d}{T_a} = \frac{VL}{\nu}  \approx \frac{\mbox{force d'inertie}}{\mbox{force de viscosité}}$
		$
		\left\{
			\begin{array}{ll}
				Re << 1 \Rightarrow \mbox{ecoulement rampant} \\
				Re \approx 1 \Rightarrow \mbox{diffusion et advection du même ordre de grandeur} \\
				Re >> 1 \Rightarrow \mbox{effets de viscosité négligeables} \\
			\end{array}
		\right.
		$
		\item \textbf{Nombre de Froude} : $Fr = \frac{V^2}{L \ g}  \approx \frac{\mbox{force d'inertie}}{\mbox{force de pesanteur}}$ 
	\end{itemize} 
\end{definition}

\begin{definition} Couche limite (approfondie par la suite) \par
	Distance à la paroie, notée $\sigma$, sur la quelle le milieu est affectée par la diffusion. 
	$$ \sigma \approx \sqrt{\nu \ T_a } \approx \sqrt{\nu \ T_d }$$ 
\end{definition}


\subsection{Rotation dans les écoulements}

\begin{definition} Theoreme de Kelvin \par
	$$ \frac{\vec{\Omega}}{2} = \frac{rot(\vec{V})}{2} $$
	\textit{Remarque} : Par le théoreme de Stockes, on obtient un lien avec la circulation
	$ \Theta(C) = \oint_C \vec{v} \ \vec{dl} = \int_S \vec{\Omega} \vec{n} \ dS$
\end{definition}


\begin{definition} Taux de rotation local \par
	Si les effets visqueux sont négligeables au voisinage d'un domaine matériel dans le fluide (fluide parfait) :
	La circulation sur un \textbf{contour matériel fermé} reste constante au cours du temps.
	\textit{Remarque} : Proche de la paroie (omniprésence des phénomene visqueux), la condition d'adhrence fait apparaitre un rotationel non nul.
\end{definition}

\begin{definition} Loi de Biot et Savart \par
	$$ \vec{V}(M) = - \int_V \frac{\vec{\Omega}(M') \wedge \vec{r}}{4 \pi r^3} dV \; \mbox{avec} \; \vec{r} = \vec{MM'} $$
	La vorticité dans l'élément de volume $dV$ \textbf{induit une rotation} de tout le fluide à la distance $r$ de vitesse de rotation $ \frac{\Omega \ dV}{4\pi r^3}$
\end{definition}

\begin{definition} Enstrophie \par
	L'enstrophie représente l'intensité tourbillonnaire (indépendement de l'orientation).
	$$ F = \frac{\Omega ^2}{2}  $$
	Seul l'étirement d'un tourbillon peut faire varier son intensé. Les autres actions se traduisent par une réorientation des filets tourbillonnaires (qui s'arretent obligatoirement sur une paroi ou sur eux-meme).
\end{definition}

\subsection{Mise en situation}

\begin{itemize}
	\item \textbf{Ecoulement de couette}

	L'echelle de temps pertinente est $ T_d = \frac{h^2}{\nu} $ pour l'établissement de la solution Couette stationnaire. Les effets visqueux sont confinés à une distance de l'ordre $\delta = \sqrt{\nu t}$ au temps $t$.
	
	\item \textbf{Ecoulement de Poiseuille en conduite}
	
	L'echelle de temps pertinente est $ T_d = \frac{a^2}{\nu} $ où $a$ est le rayon du tube. La pression est le seul role moteur. Sa force compense les fortement visqueux aux paroies et sa puissance équilibre les dissipations.

	\item \textbf{Ecoulement à lignes de courant circulaires}
	
	Le gradient radial de pression compense les effets centrifuges.

	\item \textbf{Tourbillon de Lamb-Oseen} TD 4
	
	Vitesse radiale : $ V_{\theta} = \frac{\Gamma}{2 \Pi \ r} $ où $\Gamma$ est la circulation. La longueur caractéristique de diffusion est $\Delta = \nu \ t$.

	\item \textbf{Tourbillon de Burger}, tourbillon visqueux maintenu par étirement
	
	Le rayon du coeur tourbillonnaire est approché par $ \delta_d = \sqrt{\frac{\nu}{\alpha}} = \frac{d}{Re}$ 
	avec l'étirement $\alpha = \frac{U}{d} > 0$ où $d$ est le diametre d'orifice et $U$ la vitesse débitante.

\end{itemize}

\subsection{Stabilité des écoulements}

\section{Couche limite laminaire}

\begin{definition} Présence d'une couche limite et estimation \par
	Dans les écoulements à $Re >> 1$ les effets inertielles dominent mais il existe une zone proche des paroies où la viscosité ne peut etre négligé. On peut caractéristiser l'epaisseur sur la quelle les deux effets sont présents :
	$$ T_a = T_d \Leftrightarrow \frac{L}{V} = \frac{\delta^2}{\nu} \Leftrightarrow \delta = \sqrt{\frac{\nu L}{V}} = \frac{L}{\sqrt{R_{eL}}} $$
\end{definition}

\begin{definition} \textbf{Epaisseur conventionnelle} de couche limite \par
	Soit $ U_e(x) $ la vitesse en fluide parfait, $U(x,y)$ la vitesse en considerant la viscosité.
	$$ U(x,\delta(x)) = 0.99 \ U_e(x) $$
	\textit{Remarque} : cette longueur sera utilisé dans les bilans intégraux.
\end{definition}

\begin{definition} \textbf{Epaisseur de déplacement} de couche limite \par
	Caractéristise l'écart à la situation du fuilde parfait concernant le débit à travers l'épaisseure de la couche limite.
	$$ \delta ^{*} = \int_0^{"\infty"} (1 - \frac{U}{U_e}) dy = \int_0^{\delta} (1 - \frac{U}{U_e}) dy$$
	\textit{Remarque} : $\delta^{*}$ peut etre considéré comme le déplacement de la ligne de courant $q_v = U_e(\delta-\delta^{*})$
\end{definition}

\begin{definition} \textbf{Epaisseur de quantité de mouvement} de couche limite \par
	C'est la hauteur dont il faudrait déplacer la surface de déplacement pour conserver le début de quantité de mouvement en fluide parfait.
	$$ \Theta(x) = \int_0^{\delta} \frac{U}{U_e} (1-\frac{U}{U_e}) dy $$
\end{definition}

\end{document}